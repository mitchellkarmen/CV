% this copy of my CV is meant for continuous integration.
% it's based on dan foreman-mackey's example here https://dfm.io/posts/travis-latex/
%. style and layout from david w hogg: 
%--------------------------------------------------------------
% to-do:
%   - Sections: non-research related jobs, outreach, awards, publications, presentations, code/skills, ...
%   - Make a better template
%   - Include mentorship
%   - Include SULI presentations
% REMINDER: PULL FROM GITHUB BEFORE EDITING ON OVERLEAF
%--------------------------------------------------------------

\documentclass[12pt,letterpaper]{article}

\input{cvstyle}

\begin{document}\thispagestyle{empty}\sloppy\sloppypar\raggedbottom

\textbf{\Large Mitchell Karmen} \hfill
\textsf{\small mitchell@nyu.edu} \\[0.5ex] %optional: website
Research Assistant, Supernova Cosmology Project, Lawrence Berkeley National Laboratory\\[0.5ex]
\vspace{\medskip}
I study Type Ia Supernovae, the explosions of white dwarf stars, and how they can inform us about the past, present, and future of our universe.  I combine astrophysical models and cutting-edge statistical models such as machine learning in order to develop methods of studying these events.  More generally, I'm interested in all kinds of extragalactic transients, and data science as it applies to time-domain astronomy.

\cvheading{Education}
\begin{list}{}{\cvlist}
  \item
        BA 2020, Physics, New York University.
\end{list}

\cvheading{Positions}
\begin{list}{}{\cvlist}
  \item
        Research Assistant, Lawrence Berkeley National Laboratory, 2021--present. \textit{Advisor: Saul Perlmutter}
  \item
        DOE Science Undergraduate Laboratory Intern, Lawrence Berkeley National Laboratory, 2020--2021. \textit{Advisor:Peter Nugent}
  \item
        Scientific Programmer, Zwicky Transient Facility Black Hole Working Group, Summer 2020. \textit{Advisor: Sjoert van Velzen}
\end{list}


\end{document}

