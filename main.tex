% this copy of my CV is meant for continuous integration.
% it's based on Dan Foreman-Mackey's example here https://dfm.io/posts/travis-latex/%
% -----------------------------------------
% Template From: 
% (c) 2002 Matthew Boedicker <mboedick@mboedick.org> (original author) http://mboedick.org
% (c) 2003-2007 David J. Grant <davidgrant-at-gmail.com> http://www.davidgrant.ca


\documentclass[letterpaper,10pt]{article}

%-----------------------------------------------------------
%Margin setup

\usepackage{hyperref}


\setlength{\voffset}{0.1in}
\setlength{\paperwidth}{8.5in}
\setlength{\paperheight}{11in}
\setlength{\headheight}{0in}
\setlength{\headsep}{0in}
\setlength{\textheight}{11in}
\setlength{\textheight}{9.5in}
\setlength{\topmargin}{-0.25in}
\setlength{\textwidth}{7in}
\setlength{\topskip}{0in}
\setlength{\oddsidemargin}{-0.25in}
\setlength{\evensidemargin}{-0.25in}
%-----------------------------------------------------------
%\usepackage{fullpage}
%\usepackage{shading}
%\textheight=9.0in
\pagestyle{empty}
\raggedbottom
\raggedright
\setlength{\tabcolsep}{0in}

\usepackage{enumitem}
\setlist[description]{labelwidth=0.7cm,leftmargin=0.92cm,labelindent=0cm}

%-----------------------------------------------------------
%Custom commands
\newcounter{descounter}
\newcommand{\resetcounter}[0]{\setcounter{descounter}{0}}
\newcommand{\resitem}[1]{\item #1 \vspace{-6pt}}
%\newcommand{\resheading}[1]{{\large \parashade[.9]{sharpcorners}{\textbf{#1 \vphantom{p\^{E}}}}}}
\newcommand{\resheading}[1]{\resetcounter \vspace{15pt} {\Large \textbf{#1}} \\ \vspace{-8pt}
    \hrulefill\vspace{5pt}}
\newcommand{\ressubheading}[5]{
    \vspace{10pt}
    \begin{tabular*}{7.0in}{l@{\extracolsep{\fill}}r}
        \textbf{#1} & #2 \\
        #3 & #4 \\
        #5 & \\
    \end{tabular*}\vspace{-5pt}
}
\newcommand{\descheader}[1]{\textbf{#1}}
\newcommand{\descitemold}[1]{\stepcounter{descounter} \item[\arabic{descounter}:]
    #1 \vspace{-5pt}}
\newcommand{\descitem}[2]{\stepcounter{descounter} \item[\arabic{descounter}:] \textit{#1} \\ #2\vspace{-5pt}}
%-----------------------------------------------------------

\begin{document}



\begin{center}
\textbf{\LARGE Mitchell Karmen}
\end{center}

\begin{tabular*}{7in}{l@{\extracolsep{\fill}}l}
    Johns Hopkins University & \hfill mkarmen1@jhu.edu \\
     Department of Physics and Astronomy & \hfill mitchell@nyu.edu \\
\end{tabular*}
\\

%\vspace{0.1in}

\resheading{Education}

\ressubheading{Johns Hopkins University}{Baltimore, MD}{Ph.D. Astrophysics}{Fall 2022--Present}{}{}{}

    \ressubheading{New York University}{New York, NY}{B.A. Physics}{2016--2020}{}{}{}
    
    \descheader{Additional Education}
    \vspace{-0.05in}
    
        \begin{itemize}
            \resitem{GROWTH Multi-Wavelength Astronomy Summer School \hfill 2020 } \\
            \resitem{Center for Computational Astrophysics Data Analysis Bootcamp \hfill 2018--2019} \\
        \end{itemize}

\resheading{Research Positions}

    \ressubheading{Transient Science @ Space Telescope Science Institute}{Graduate Research Assistant}{Advisor: Suvi Gezari}{Fall 2022--Present \\
    \textit{Detecting tidal disruption events in the early universe}}
    
    \ressubheading{Supernova Cosmology Project, Lawrence Berkeley National Laboratory}{Berkeley, CA}{Research Assistant}{Spring 2021--Summer 2022}{Advisor: Saul Perlmutter \\ \textit{Standardizing Type Ia Supernova Spectral Time Series with Neural Networks}}
    
        % \begin{itemize}
        %         \resitem{Developed new methods of standardizing Type Ia
        %             Supernovae (SNe Ia) using Neural Networks built in PyTorch and Gaussian Processes in George.}
                    
        %         \resitem{Built a custom SNe Ia spectral time-series fitter, and generative model to be used in time-domain survey simulations.}
        % \end{itemize}
    
    \ressubheading{Center for Computational Cosmology}{Berkeley, CA}{DOE Science Undergraduate Laboratory Intern}{Fall 2020--Spring 2021}{Advisor: Peter Nugent \\
    \textit{Using convolutional neural networks to investigate SN Ia detonation mechanisms}}
    
        % \begin{itemize}
        %         \resitem{Performed image subtractions of early-detection Zwicky Transient Facility SNe Ia using MPI and the high performance computing capabilities of NERSC.}
        %         \resitem{Used Convolutional Autoencoder Neural Networks built in Tensorflow to cluster Type Ia multi-band light curves for cosmology analysis.}
        % \end{itemize}
    
    \ressubheading{Black Hole Working Group, Zwicky Transient Facility}{New York, NY}{Scientific Software Developer}{Summer 2020}{Advisor: Sjoert van Velzen \\
    \textit{Main programmer for the ZTF tidal disruption event detection and analysis pipeline}}
    
        % \begin{itemize}
        %         \resitem{Developed the software for the ZTF Tidal Disruption Event alert pipeline in DESY's Ampel framework.}
        %         \resitem{Wrote algorithms for nightly TDE candidate detection, analysis, potential classification, and follow-up.}
        % \end{itemize}
    
    \ressubheading{NYU Center for Cosmology and Particle Physics}{New York, NY}{Undergraduate Researcher}{2019--2020}{Advisors: Glennys Farrar, Sjoert van Velzen \\
    \textit{Creating an all-sky catalog of compact galaxies}}
    
        % \begin{itemize}
        %         \resitem{Cross-matched the \textit{Gaia} and 2MASS surveys to develop a novel metric for galaxy central compactness.}
        %         \resitem{Compiled an all-sky catalog of compact galaxies for use in nuclear transient follow-up.}
        % \end{itemize}

\clearpage

\resheading{Publications}

\begin{itemize}
        \item{ \textit{A Probabilistic Autoencoder for Type Ia Supernovae Spectral Time Series} \\Stein, G., et al., ApJ (2022) }
        
        \item{\textit{Bump Morphology of the CMAGIC Diagram} \\ Aldoroty, L., et al., Submitted to AJ (2022)}

\end{itemize}

\descheader{In Preparation}
    
    \begin{itemize}
    
        \item{ \textit{Extending the Twins Embedding to Correct Type Ia Supernova Peak Magnitudes with Single Spectra at Any Phase} \\ \textbf{Karmen, M.}, The Nearby Supernova Factory, In Preparation to submit to ApJ (2023) }
        \vspace{-5pt}
        
    \end{itemize}

\resheading{Awards and Grants}
    
    \ressubheading{George Granger Brown Award}{2020}{NYU Department of Physics}{}{}{}
    
    \vspace{-0.15in}
    \ressubheading{Dean's Conference Grant}{2020}{NYU Dean's Undergraduate Research Fund}{}{}{}
    
    \vspace{-0.15in}
    \ressubheading{Collegiate Research Scholar}{2019}{NYU College of Arts and Sciences}{}{}{}
    
    \vspace{-0.15in}
    \ressubheading{DURF Grant}{2019}{NYU Deans Undergraduate Research Fund}{}{}{}
    
    \vspace{-0.15in}
    \ressubheading{Research+ Scholar}{2018}{NYU College of Arts and Sciences}{}{}{}
    \vspace{-0.15in}

\resheading{Outreach and Leadership}

    \ressubheading{Volunteer Mentor}{2021--Present}{Berkeley Lab Director's Apprenticeship Program}{}{}
    \vspace{-0.15in}
    
    \ressubheading{Mentor}{2021}{Tampa Heights Junior Civic Association}{}{}
    \vspace{-0.15in}
    
    \ressubheading{NYU Society of Physics Students}{2019--2020}{President}{}{}{}
    \vspace{-0.15in}
    
    \ressubheading{Volunteer Instructor}{2019--2020}{BioBus}{}{}
    \vspace{-0.15in}
    
    \ressubheading{Lead Organizer}{2019--2020}{NYU Physics Mentorship Program}{}{}
    \vspace{-0.15in}
    
    \ressubheading{NYU Society of Physics Students}{2017--2018}{Treasurer}{}{}{}
    \vspace{-0.15in}

\resheading{Conferences and Talks}

\resetcounter

\begin{description}
    \descitem{Extending the Twins Embedding for Use by the Roman/WFIRST Supernova Survey}
        {Contributed Talk, Roman Science Team Community Briefing, Virtual (2021)}
        
    \descitem{Investigating the Variability of Type Ia Supernovae Using Deep Learning}
    {Poster, LBNL Spring Intern Poster Session, Virtual (2021)}
    
    \descitem{Unsupervised Clustering of Type Ia Supernova Light Curves}
    {Poster, LBNL Fall Intern Poster Session, Virtual (2020)}
    
    \descitem{A New All-Sky Catalog of Compact Galaxies}
    {Poster, American Astronomical Society Meeting 235, Honolulu, HI (2020)}
    
    \descitem{A New All-Sky Catalog of Compact Galaxies: Opportunities for Electromagnetic Transient Follow-Up}{Poster, Gotham AstroFest, New York, NY (2019)}
    
    \descitem{Finding Exoplanets in Strange Orbits}{Contributed Talk, NYU Undergraduate Research Symposium, New York, NY (2018)}
 

\end{description}

\resheading{Teaching}

    \ressubheading{Teaching Assistant}{2022--2023}{Johns Hopkins University}{}{}
    \vspace{-0.15in}

    \begin{itemize}
        \resitem{Planets, Life, and the Universe}
        \resitem{General Physics Lab I}
        \resitem{General Physics Lab II}
    \end{itemize}

    \ressubheading{Departmental Tutor}{Fall 2018--Spring 2020}{New York University}{}{}
    \vspace{-0.15in}
    
        \begin{itemize}
                \resitem{General Physics I: Kinematics}
                \resitem{General Physics II: Electricity and Magnetism}
        \end{itemize}
    
    \ressubheading{7th Grade Mathematics Classroom Tutor}{Fall 2017--Spring 2018}{America Reads / America Counts}{}{}
    \vspace{-0.15in}
    
        \begin{itemize}
            \resitem{Assisted a middle school math teacher with classroom activities}
            \resitem{Tutored one-on-one with students struggling in mathematics}
        \end{itemize}
    
    % \ressubheading{Mathematics Tutor}{2014--2016}{Mathnasium}{}{}
    % \vspace{-0.15in}
    
    %     \begin{itemize}
    %         \resitem{Led group after-school tutoring sessions of mathematics subjects up through calculus.} 
    %     \end{itemize}


\end{document}

